%----------------------------------------------------------------------------------------
%	PACKAGES AND OTHER DOCUMENT CONFIGURATIONS
%----------------------------------------------------------------------------------------

\documentclass[twoside]{article}

\usepackage{lipsum} % Package to generate dummy text throughout this template

\usepackage[sc]{mathpazo} % Use the Palatino font
\usepackage[T1]{fontenc} % Use 8-bit encoding that has 256 glyphs
\linespread{1.05} % Line spacing - Palatino needs more space between lines
\usepackage{microtype} % Slightly tweak font spacing for aesthetics

\usepackage[hmarginratio=1:1,top=32mm,columnsep=20pt]{geometry} % Document margins
\usepackage{multicol} % Used for the two-column layout of the document
\usepackage[hang, small,labelfont=bf,up,textfont=it,up]{caption} % Custom captions under/above floats in tables or figures
\usepackage{booktabs} % Horizontal rules in tables
\usepackage{float} % Required for tables and figures in the multi-column environment - they need to be placed in specific locations with the [H] (e.g. \begin{table}[H])

\usepackage{lettrine} % The lettrine is the first enlarged letter at the beginning of the text
\usepackage{paralist} % Used for the compactitem environment which makes bullet points with less space between them

\usepackage{abstract} % Allows abstract customization
\renewcommand{\abstractnamefont}{\normalfont\bfseries} % Set the "Abstract" text to bold
\renewcommand{\abstracttextfont}{\normalfont\small\itshape} % Set the abstract itself to small italic text

\usepackage{titlesec} % Allows customization of titles
\renewcommand\thesection{\Roman{section}} % Roman numerals for the sections
\renewcommand\thesubsection{\Roman{subsection}} % Roman numerals for subsections
\titleformat{\section}[block]{\large\scshape\centering}{\thesection.}{1em}{} % Change the look of the section titles
\titleformat{\subsection}[block]{\large}{\thesubsection.}{1em}{} % Change the look of the section titles

\usepackage{fancyhdr} % Headers and footers
\pagestyle{fancy} % All pages have headers and footers
\fancyhead{} % Blank out the default header
\fancyfoot{} % Blank out the default footer
\fancyhead[C]{Social Psychology $\bullet$ Spring Semester 2014 $\bullet$ Special Assignment 1} % Custom header text
\fancyfoot[RO,LE]{\thepage} % Custom footer text
\newcommand{\pad}{Partially Analysed Data }

%----------------------------------------------------------------------------------------
%	TITLE SECTION
%----------------------------------------------------------------------------------------

\title{\vspace{-15mm}\fontsize{24pt}{10pt}\selectfont\textbf{Social Construction: Happiness}} % Article title


\author{
\large
\textsc{Srijan R. Shetty} \\ % Your name
\normalsize{11727}\\[2mm] % Roll Number
\normalsize{Department of Computer Science and Engineering,} \\ % Department
\normalsize Indian Institute of Technology, Kanpur \\ % Your institution
\normalsize {srijans@iitk.ac.in} % Your email address
\vspace{-5mm}
}
\date{}

%----------------------------------------------------------------------------------------

\begin{document}

\maketitle % Insert title

\thispagestyle{fancy} % All pages have headers and footers

%----------------------------------------------------------------------------------------
%	ABSTRACT
%----------------------------------------------------------------------------------------

\begin{abstract}

\noindent 
This paper describes an experiment conducted in Social Psychology class to understand
social construction as a paradigm to comprehend subjective experiences between humans.
The objective of this paper is to highlight how the researcher's definition of a subjective
experience --- happiness --- undergoes an evolution over the course of the experiment.

\end{abstract}

%----------------------------------------------------------------------------------------
%	ARTICLE CONTENTS
%----------------------------------------------------------------------------------------

\begin{multicols}{2} % Two-column layout throughout the main article text

\section{Introduction}

\lettrine[nindent=0em,lines=3]{S} ocial Construction is a contemporary paradigm to 
comprehend subjective experiences. Subjective experience by their virtue show a lot
of variability and contemporary methods fail to capture the richness of these experiences.
In social construction, unlike contemporary methods, the researcher starts with his own
definition of the experience which gets moulded into something new during the course of the
experiment.\\
In this experiment, a class is unknowingly pushed into the journey of understanding 
social construction by experimentation. The results have been compiled at the end of the experiment.

%------------------------------------------------

\section{Methodology}
The following section describes the methodology which was followed in the experiment.

\subsection{Control}
The control of the experiment was established by asking each of the students to pen down
what happiness meant to each of them. They were given no a priori information about the
objective of the experiment, in order to ensure that their definitions were not influenced
by any extraneous factors.
The students were advised to keep this definition until the very end of the experiment.

\subsection{Dialogical Partnership}
The second phase of the experiment, involved setting up of a dialogical partnership between
pairs of students. The students were given complete freedom to form pairs; they were then asked
to exchange the definitions that they had penned down. It must be noted that while most students
paired up with friends, some paired up with complete strangers.
The initial exchange followed by an assessment of how well they comprehended each other's definition.

\subsection{Partial Analysis}
In order to broaden the scope of the experiment, the students were asked to form groups of eight. 
These groups were then asked to analyse the definitions of the individual members of the groups
and the dialogical partnership to come up the commonalities and differences in the definitions
of the individual members.
The groups were to also appoint a group leader who would present a consolidated analysis to the 
entire class and compile a document containing the inferences.

\subsection{Individual Analysis}
After the presentations, an elaborate discussion ensued which explained the purpose of all the
exercises. The students were then asked to analyse each of the nine (the number of groups were nine)
partially analysed data to come up with a consolidated overall report.
The students were also asked to report whether or not their definition of happiness evolved over the
course of this experiment.

%------------------------------------------------

\section{Analysis}
A salient point that I would like to put forth before I proceed with the analysis is, the parallel I could draw
between the data obtained by social constructionist approach and the data obtained by any conventional 
While at the surface, the data seems to be discordant, when you move past the surface cacophony, you 
start seeking a beautiful similarity.
Even subjective experiences such as \emph{happiness}, which I vehemently believed to be a function of
various factors like the individual's upbringing, socio-economic status etc., have an underlying
uniformity.

The following section describes the similarities and differences that could be clearly distinguished from
the partially analysed data obtained from the individual groups.

\subsection{Similarities}

\noindent \emph{Relationships}\\
The \pad gives clear evidence that spending quality time with friends and kin is the most common source of
happiness for many. This can be easily seen as a consequence of our social nature, we like to be understood
by our close ones. From playing a sport with friends, to sitting and playing poker with family, watching movies,
playing video games, eating food after a hectic day with you parents, all are myriad ways in which one spends
quality time with friends and family.\\

\noindent \emph{Accomplishments}\\
Evidence suggests that excelling in the endeavours that one participates in makes them happy. Again, the
magnitude of the task shows a lot of variability: from solving a simple mathematics problem to
cracking a tough examination like IITJEE, from winning in a video game to winning high stakes poker, the \pad 
has it all.
A salient point to notice is the fact that while some people excel for their own selfish reasons, many excel 
for their parents or their near and dear ones for they believe that their excellence makes their loved ones 
happy. Again this was something which I found very hard to digest because I vehemently believe that you
should pursue any interest only out of your selfish interest for otherwise, your love for that interest is a 
facade.
While my inference may not be true, there is enough evidence to suggest that being accomplished in life
makes one likeable in social circles; and likeability is desired by human beings because they want to belong and
not feel ostracised in life.\\

\noindent \emph{Satisfaction}\\
Satisfaction as a source of happiness seems logical, and the \pad clearly shows that that people are happy
if they are satisfied in their professional or academic life. What really amused me was the fact
while some people are satisfied with being average and with a lacklustre life, there are some whose thirst
cannot be satiated even after excelling in everything they take part in. \\
%===============================ADD==========================

\noindent \emph{Stress free environment}\\
Almost everyone agreed to the fact that they were happy if they were placed in a stress
free environment where they were free to chose what they wanted to do: be it slacking off and sleeping or 
relaxing by watching a movie or a TV series, or just free to pursue their hobbies in an environment
devoid of the anxiety of quizzes and examinations. 
This could be because of the reason that we are better able to enjoy the activities that we take part
in if we are not anxious and bogged down by the many fears that we have in life. If there is no stress
we are able to concentrate more at the task at hand and perform better at it, accomplishing in it as
well as being satisfied with the work that we have done.

\subsection{Differences}

\noindent \emph{Short term vs long term happiness}\\
There were several instances which pointed out the difference between momentary joy and long term happiness.
Examples included winning in a poker game or video game which were labelled as momentary joy. What is to be
noted is that in such a distinction of joy and happiness, we inadvertently make happiness into an
aspiration of how we want out future to be. And in many sense, this strips down happiness of its rich
meaning because our goals keep on changing, and we always keep setting new goals.
I personally believe that distinguishing momentary joy from happiness, strips down the rich meaning of
happiness; and that happiness is living in the moment. \\

\noindent \emph{Individual vs. Collective}\\
The Individual vs. Collective discourse appears in the \pad in many different avatars. This particular theme
has been brought to light under different banners in the \pad: patriotism, social responsibility, and group 
victory. This whole discourse boils down a simple question: whether the individual thinks of happiness as 
a personal construct or that he thinks that happiness is a shared construct. In the former case, only a personal
achievement can bring about happiness for the individual while in the latter, happiness can be induced by any
group feeling as well.\\

\noindent \emph{Abstract vs. Concrete}\\
The abstract vs. concrete debate was visible in the \pad. This particular point overlaps with almost every
other point made in this discourse for the different themes as listed in this discourse can be labelled as 
abstract or concrete. 
Many people attributed happiness to abstract entities like aesthetic beauty, satisfaction, accomplishment
while on the hand there were quite a few who attribute happiness to concrete entities like owing a material
object, learning a new skill, eating good food.\\

\noindent \emph{Lifestyle Preferences}\\
Lifestyle preferences could be attributed to the way the individuals have been brought up. While some said
that a profligate lifestyle which allowed them to travel, to eat good food and shop for new clothes and shoes
made them happy, others felt that they found no happiness in such materialistic pursuits.
On similar lines, there were people who believed that they couldn't be happy with a mundane stable life with 
little or not change and only adventure could bring them happiness; and there were people with an antithetical
view of happiness.
What this difference proves is the fact, that the intuitive idea that happiness does have something to do
with the upbringing of the individual is indeed true.

%------------------------------------------------

\section{Individual Analysis}

\subsection{Control}
The initial definition of happiness that I came up with was:
\begin{quotation}
    Happiness for me revolves around two extremes, the first is happiness in solitude and the second one is with
    friends. What I mean to say that I do not mind being alone writing poetry, reading up about something
    interesting, or learning a new skill. At the same time, I do not mind hanging out with friends and spending some
    quality time with them.
\end{quotation}

\subsection{Dialogical Partnership}
In the Dialogical Partnership exercise, I was fortunate enough to find a friend to pair up with.
This made the entire discourse very easy for both of us, despite the fact that our definitions
of happiness were antagonistic of each other.
While a stress free environment where Smita could sleep whenever he wanted to, play poker at
at lengths --- defeat people at poker to be precise -- without being disturbed, watch movies
and relax made him happy.
I preferred to invest my time in what I believe to be the pursuit of knowledge: writing poems, reading,
and acquiring new skills. But this did not mean that we did not have any similarities, both of 
us loved spending quality time with friends or watching a good movies. 
What confounded me  was the fact that despite having colliding definitions of happiness, Smita 
was able to empathize with my definition (which could be attributed to our closeness).

\subsection{Partial Analysis}
The group analysis stage brought to light many of the shortcomings of the dialogical partnerships
that were established especially those between strangers. To list a few:

\begin{compactitem}
\item The strangers had inhibitions while sharing their definitions of happiness, because they felt 
    that they would be judged.
\item The members acted as sycophants and agreed to each others point even if they did not relate to them.
\item The pairs who knew each other from before were better at empathizing with each others definitions.
\item The definition was dependent on the context it was asked and the person who asked the question.
\item There were certain facets of the definition of happiness, that strangers could not share between each other.
\end{compactitem}

Being the group leader I was burdened with the additional responsibility of compiling the similarities and the
differences of my group. This gave me time to analyse the data very carefully and realize the gaping holes that
I had in my definition of happiness. I could clearly see that the initial definition that I came up with was 
not exhaustive and had some gaping holes like the fact that

\begin{compactitem}
\item watching TV series, movies and listening to good music made me happy.
\item travelling to new places and buying gadgets and new shoes also made me happy.
\end{compactitem}

My definition expanded to include materialistic things --- gadgets and shoes ---, abstract concepts --- travelling
and exploring new places ---, relaxing --- watching TV series and listening to music. So the entire exercise,
enriched my definition of happiness and made me realize how co-construction could help you realize more about 
yourself.

\subsection{Individual Analysis}
The individual analysis made me realize some clear distinctions between my definition of happiness and the 
general definition of happiness. Unlike the general definitions, my desire to learn something new is 
motivated completely by selfish reasons. At the same time I realized that the its the details and little 
things that matter more to me when it comes to happiness. Being alone and spending time with friends at
first seem to be antagonistic to each other but boil down to the fact that I like living in the moment, 
something which I could not have concluded without the help of this experiment.

%------------------------------------------------

\section{Unanswered Questions}
There are many unanswered questions that need to be addressed by further experimentation:\\

\noindent \emph{How does the definition of happiness of an individual change over time? Is there something 
fundamental in the definition that remains constant or can it alter completely?}\\
Experimentation is a bit difficult in this particular case because the studies will last for years. But,
it can be interesting if we can know about the nature of happiness in individuals over time.\\

\noindent \emph{How does one cope with a constraint which can potentially curb their happiness? Does one adapt or 
perish?}\\
A particular example of such a situation is when one is faced with the harshities of life like sudden
unemployment, how do they cope with such a situation.

%------------------------------------------------

\section{Conclusion}
This experiment started off as a forced exercise and in the end lead to a self realization about the nature of
happiness. In retrospect, I would have never been able to realize the many facets of the definition of 
happiness if it weren't for this experiment. Also. I've come to the conclusion that subjective experiences
evolve with time and we imbibe a lot of the qualities that we see around ourselves. \\
As a consequence of this experiment, I have been trying to see whether some of the popular themes which
came up in the differences could somehow make me happy in a different context.

\end{multicols}

\end{document}
